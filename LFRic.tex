\documentclass[times]{elsarticle}
\bibliographystyle{elsarticle-num}
\usepackage{graphicx,colordvi}
\begin{document}
\begin{frontmatter}

\title{LFRic and PSyclone: Building a Domain Specific Language for Weather and Climate models}

\author[met]{S.~Adams}
\author[hartree]{R.~Ford}
\author[met]{M.~Hambley}
\author[met]{J.M.~Hobson}
\author[met]{I.~Kavcic}
\author[met,read]{C.~M.~Maynard}
\ead{c.m.maynard@reading.ac.uk}
\author[met]{T.~Melvin}
\author[bath]{E.H.~Mueller}
\author[met]{S.~Mullerworth}
\author[hartree]{A.~Porter}
\author[downunder]{M.~Rezney}
\author[met]{B.~Shipway}
\author[met]{R.~Wong}




\address[met]{Met Office, FitzRoy Road, Exeter, EX1 3PB}
\address[read]{Department of Computer Science, Polly Vacher Building,
  University of Reading, Reading, UK, RG6 6AY}
\address[bath]{Department of Mathematics, University of Bath, Bath}
\address[downunder]{University of Monash, Melbourne, Australia}
\address[hartree]{Hartree Centre, STFC Daresbury, Grim up North}

\begin{abstract}
\end{abstract}

\begin{keyword}
solvers
\end{keyword}

\end{frontmatter}

\section{Introduction}
In common with many science applications, Exascale computing presents
a disruptive change for weather and climate models. However, the
difficulty in porting and optimising legacy codes to new hardware is
particular acute for this domain as the software is large ($O(10^6)$
lines of code), takes a long time develop ($\sim 10$ years for a new
dycore) and is long-lived (typically $\sim 25$ years or longer). These
timescales are much longer than the changes in both processor
architectures and programming models necessary to exploit these
architectures~\cite{gmd-2017-186}. Moreover, highly scalable
algorithms are necessary to exploit the necessary degree of
parallelism Exascale computers are likely exhibit.

In collaboration with academic partners, the Met Office is developing
a new dynamical core, called Gung Ho~\cite{MELVIN2018342}. By employing a
mixed Finite Element Method on an unstructured mesh, the new dycore is
designed to maintain the scientific accuracy of the current Unified Model (UM)
dycore (ENDGame~\cite{QJ:QJ2235}), whilst allowing improved scalability
by avoiding the singular pole of the lon-lat grid. A new atmospheric
model and software infrastructure, named LFRic after Lewis Fry
Richardson, is being developed to host the Gung Ho dycore, as the the structured, lon-lat
grid is inherent in the data structures of the UM.

The software design is based on a layered architecture and a 
{\em separation of concerns} between the natural science code in which
the mathematics is expressed and the computational science code where the
parallelism and other performance optimisations are expressed. In particular,
there are three layers. The top layer, the algorithm layer, is where high-level mathematical 
operations on global fields are performed. The bottom layer is the kernel layer
where these operations are expressed on a single column of data. In between is the
Parallelisation System or PSy layer, where the horizontal looping and parallelism is
expressed. This abstraction, called PSyKAl, is written in Fortran with Fortran 2003
Object Orientation to encode the rules of the API.
Moreover, a Python code called Psyclone can parse the algorithm and kernel layers and
generate the Psy layer with different target programming models. In effect, the PSyKAl API
and Psyclone are a Domain Specific Embedded Language (DESL). Natural science code which
conforms to this API can be written in serial and the parallel code is then generated automatically.

The model is under active development and indeed, the science and
numerical analysis are still areas of active research. However, in
order to assess the scientific performance of the model, sufficiently
computationally challenging problems must be tackled. Thus, the
ability to generate optimisations for current architectures is also
required. In this paper, the software design and strategy for future
architectures is presented. Moreover, some preliminary performance
analysis is presented, including scaling analysis to a quarter of
million cores on the Met Office Cray XC40. The use of redundant
computation, shared memory threaded parallelism such as OpenMP and
OpenACC and performance on different architectures such as CPUs, GPUs
and ARM processors are also discussed. Furthermore, as I/O is a significant performance
factor for weather and in particular climate models some analysis of
I/O at scale using the asynchronous IO server of the XIOS library is
presented.

\section{\label{sec:GH}Gung Ho}
A bit about Gung Ho, but mostly the unstructured mesh, quads, 2+1D
mesh, vertically structured. Etc.

\section{\label{sec:SoC}Separation of Concerns}
F2K3, design, libraries, ESFM and YAXT, XIOS etc. Separation of
Concerns, PSyKal and the PSy layer

\section{\label{sec:Pscylone}PSyclone}.
The code generator.

\section{\label{sec:scal}Scaling}.
Scaling, MPI and OpenMP etc.

\section{Conclusion}
\label{sec:con}
Conclusions, future work etc


\bibliography{mibib.bib}

\end{document}
