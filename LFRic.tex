\documentclass[times]{elsarticle}
\bibliographystyle{elsarticle-num}
\usepackage{graphicx,colordvi}
\begin{document}
\begin{frontmatter}

\title{LFRic and PSyclone: Building a Domain Specific Language for Weather and Climate models}

\author[met]{S.~Adams}
\author[hartree]{R.~Ford}
\author[met]{M.~Hambley}
\author[met]{J.M.~Hobson}
\author[met]{I.~Kavcic}
\author[met,read]{C.~M.~Maynard}
\ead{c.m.maynard@reading.ac.uk}
\author[met]{T.~Melvin}
\author[bath]{E.H.~Mueller}
\author[met]{S.~Mullerworth}
\author[hartree]{A.~Porter}
\author[downunder]{M.~Rezny}
\author[met]{B.~Shipway}
\author[met]{R.~Wong}




\address[met]{Met Office, FitzRoy Road, Exeter, EX1 3PB}
\address[read]{Department of Computer Science, Polly Vacher Building,
  University of Reading, Reading, UK, RG6 6AY}
\address[bath]{Department of Mathematics, University of Bath, Bath}
\address[downunder]{University of Monash, Melbourne, Australia}
\address[hartree]{Hartree Centre, STFC Daresbury, Grim up North}

\begin{abstract}
\end{abstract}

\begin{keyword}

\end{keyword}

\end{frontmatter}

\section{Introduction}
In common with many science applications, Exascale computing presents
a disruptive change for weather and climate models. However, the
difficulty in porting and optimising legacy codes to new hardware is
particular acute for this domain as the software is large ($O(10^6)$
lines of code), takes a long time develop ($\sim 10$ years for a new
dynamical core) and is long-lived (typically $\sim 25$ years or longer). These
timescales are much longer than the changes in both processor
architectures and programming models necessary to exploit these
architectures~\cite{gmd-2017-186}. Moreover, highly scalable
algorithms are necessary to exploit the necessary degree of
parallelism Exascale computers are likely exhibit.

In collaboration with academic partners, the Met Office is developing
a new dynamical core, called Gung Ho~\cite{MELVIN2018342}. By
employing a mixed Finite Element Method on an unstructured mesh, the
new dynamical core is designed to maintain the scientific accuracy of
the current Unified Model (UM) dycore (ENDGame~\cite{QJ:QJ2235}),
whilst allowing improved scalability by avoiding the singular pole of
the lon-lat grid. A new atmospheric model and software infrastructure,
named LFRic after Lewis Fry Richardson, is being developed to host the
Gung Ho dynamical core, as the the structured, lon-lat grid is inherent in the
data structures of the UM.

The software design is based on a layered architecture and a 
{\em separation of concerns} between the natural science code in which
the mathematics is expressed and the computational science code where the
parallelism and other performance optimisations are expressed. In particular,
there are three layers. The top layer, the algorithm layer, is where high-level mathematical 
operations on global fields are performed. The bottom layer is the kernel layer
where these operations are expressed on a single column of data. In between is the
Parallelisation System or PSy layer, where the horizontal looping and parallelism is
expressed. This abstraction, called PSyKAl, is written in Fortran with Fortran 2003
Object Orientation to encode the rules of the API.
Moreover, a Python code called Psyclone can parse the algorithm and kernel layers and
generate the Psy layer with different target programming models. In effect, the PSyKAl API
and Psyclone are a Domain Specific Embedded Language (DESL). Natural science code which
conforms to this API can be written in serial and the parallel code is then generated automatically.

The model is under active development and indeed, the science and
numerical analysis are still areas of active research. However, in
order to assess the scientific performance of the model, sufficiently
computationally challenging problems must be tackled. Thus, the
ability to generate optimisations for current architectures is also
required. 

The paper is organised as follows, the Gung Ho dynamical core and
computational aspects are presented in section~\ref{sec:GH}. The
software design for the separation of concerns and PSKAl API are described 
are described in section~\ref{sec:SoC}. The model infrastructure and
use of libraries is discussed in section~\ref{sec:lib}. PSyclone, the
code generator is presented in section~\ref{sec:Psyclone}. Finally a
scaling analysis is presented in section~\ref{sec:scal} and
conclusions drawn in section~\ref{sec:con}.

\section{\label{sec:GH}Gung Ho}
A bit about Gung Ho, but mostly the unstructured mesh, quads, 2+1D
mesh, vertically structured. Etc.

\section{\label{sec:SoC}Separation of Concerns}
Science applications in general and weather and climate codes in
particular are written in high-level languages such as Fortran or
C/C++. Fortran is commonly employed for weather and climate codes and
can be considered a Domain Specific Language (DSL) for numerical
computation.  The code is structured to solve a mathematical problem
as an alogorithm and in principle, separate from hardware architecural
concerns. The compiler generates machine specific instructions from
code which conforms to the langauge standard and can, in principle,
make optimisation choices to exploit the architecture of different
processors. 

This abstraction of a separation of concerns between mathematics code
and machine code is powerful and allows for both performance and
portabilty of science codes.

\section{\label{sec:lib}Infrastructure}.
ESMF, YAXT, XIOS, pFunit? netcdf etc.

\section{\label{sec:Psyclone}PSyclone}.
The code generator.

\section{\label{sec:scal}Scaling}.
Scaling, MPI and OpenMP etc.

\section{Conclusion}
\label{sec:con}
Conclusions, future work etc


\bibliography{mibib.bib}

\end{document}
